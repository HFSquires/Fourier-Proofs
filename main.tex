\documentclass{article}
\usepackage{defaultpackages}

\hypersetup{pdftitle={Harrison's Fourier Proofs},pdfpagemode=FullScreen}




\title{Fourier Proofs}
\author{Harrison Squires - s2191671}
\date{November 2023}


\begin{document}

\maketitle
\thispagestyle{empty}



\tableofcontents
\thispagestyle{empty}



\newpage
\clearpage
\pagenumbering{arabic} 
\section{Important definitions for the course}
\begin{itemize}
\item
The complex basis function is defined as:
\begin{equation*}
    \phi_m (x)=e^{ik_mx}
\end{equation*}
\item 
Integral of dirac-delta:
\begin{equation*}
\intinf \delta(x - a) \, dx = \begin{cases} 1, & \text{if } x = a \\ 0, & \text{otherwise} \end{cases}
\end{equation*}

\item 
Integral form of dirac-delta:
\begin{equation} \label{eq:IFDD}
    \delta(x-a) = \frac{1}{2\pi}\intinf e^{ik(x-a)}\;dk
\end{equation}

\item
Wave number defined as:
\begin{equation*}
    k_n = \frac{n\pi}{L}
\end{equation*}

\end{itemize}
\newpage
\section{Derivation of complex fourier coefficients}
\begin{align}
    f(x) &= \sum_{n=-\infty}^\infty c_n\exp(ik_nx) \label{complexcoef}\\ 
    \implies \int_{-L}^L f(x)\exp(-ik_mx)\;dx&=\int_{-L}^L \exp(-ik_mx)\sum_{n=-\infty}^\infty c_n\exp(ik_nx)\;dx \label{complexcoef2}\\
    \implies  \int_{-L}^L f(x)\exp(-ik_mx)\;dx&=\sum_{n=-\infty}^\infty c_n\int_{-L}^L \exp(-ik_nx)\exp(ik_nx) \; dx\\
    \implies  \int_{-L}^L f(x)\exp(-ik_mx)\;dx&=c_n 2L\delta_{mn}\\
    \implies c_n  &= \frac{1}{2L}\int_{-L}^Lf(x)\exp(-ik_nx)\;dx \label{complexcoefflast}
\end{align}

Here between Eq. \ref{complexcoef} and Eq. \ref{complexcoef2} both sides have been multiplied by $\exp(-ik_mx)$ and integrated with respect to $x$ in the boundary $\left [-L ,L\right ]$. In Eq. \ref{complexcoefflast} the index is swapped to n using property of the Kronecker-delta, $\delta_{mn}$.

\newpage
\section{Proof of orthogonality relation (complex)} \label{sec:ortho}
\begin{align*}
    \int_{-L}^{L} dx \;\phi_m \phi_n^* &= \int_{-L}^{L} dx \;e^{i(k_m-k_n)x}\\
    &=\begin{cases}
      \left [x \right]^L_{-L}, & \text{if}\ m=n \\
      \frac{1}{i(k_m-k_n)}e^{i(k_m-k_n)x}, & \text{if}\ m\neq n
    \end{cases} \\
    &=\begin{cases}
      2L, & \text{if}\ m=n \\
      0, & \text{if}\ m\neq n
    \end{cases}\\
    &= 2L\delta_{mn}
\end{align*}
This is the defintion of orthogonality, the product of two functions integrated over some space evaluates to 0, if they are mutually orthogonal.

\newpage
\section{Proof of convolution theorem}
The convolution theorem is:
\begin{equation}
    f(x)*g(x)=\tilde{f}(k)\tilde{g}(k)
\end{equation}
The definition of a convolution is: 
\begin{equation}
    f(x)*g(x) = \intinf dy\;f(x-y)g(y)
\end{equation}
The arguments of f \& g can be swapped round, this is not intuitive but true.
\begin{align*}
    f*g &= \intinf f(y)g(x-y) \\
    f(y)&= \frac{1}{2\pi}\intinf\FT{f}e^{-iky}\;dk \\
    g(x-y)&=\frac{1}{2\pi}\intinf\FTp{g}e^{-ik{'}(x-y)}\;dk{'} \label{primed} \\
    \implies f*g &= \frac{1}{(2\pi)^2} \intinf dy \intinf\FT{f}e^{-iky}\;dk\intinf\FTp{g}e^{-ik{'}(x-y)}\;dk{'}\\
    &=\frac{1}{(2\pi)^2}\intinf dk\; \FT{f}\intinf dk{'} \FTp{g} e^{ik{'}x} 
    \left ( \intinf e^{-i(k-k')y} dy \right )\ \\
    \implies \left ( \intinf e^{-i(k-k')y} dy \right ) &= 2\pi \delta(k-k') = 2\pi \delta_{kk'}\ \color{blue}\leftarrow\text{(Using the integral form of the dirac-delta , Eq.\ref{eq:IFDD})}\\
    \implies f*g &= \frac{1}{2\pi} \intinf \FT{f}\;dk\left ( \intinf \FTp{g} e^{ik'x} \delta_{kk'} dk' \right)\\
    &= \frac{1}{2\pi} \intinf \FT{f}\FT{g}e^{ikx}dk \\
    \implies \widetilde{f*g} &= \FT{f}\FT{g}  
\end{align*}
$k{'}$ is used as opposed to $k$ to show the two integrals are different.

\newpage
\section{Proof of Parseval's theorem using integral form of Dirac-delta}

    Parseval's theorem states:
    \begin{align*}
    \intinf |f(x)|^2 \, dx &= \intinf |\FT{f}|^2 \, dk \\
    &= \intinf f(x) \cdot f^*(x) \, dx \\
    &= \intinf dx \,\left(\frac{1}{2\pi} \intinf \FT{f} \, e^{ikx} \, dk\right) \left(\frac{1}{2\pi} \intinf \FTp{f^*}\, e^{-ik'x} \, dk'\right)  \\
    &= \frac{1}{(2\pi)^2} \intinf \intinf \FT{f} \, \FTp{f^*} \, 2\pi \, \delta(k - k') \, dk \, dk'\ \color{blue}\leftarrow\text{(Using Eq.\ref{eq:IFDD})} \\
    &= \frac{1}{2\pi} \int_{-\infty}^{\infty} |\tilde{f}(k)|^2 \, dk
    \end{align*}
Here the defintion of the FT has been applied to both $f(x)$ and $f^*(x)$.

\newpage
\section{FT of exponentially decaying function}
Let:
\begin{equation*}
    h(t) = e^{-a|t|}, \ \text{where } a > 0
\end{equation*}
The fourier transform of $h(t)$, $\FT{h}$ is given by:
\begin{equation*}
    \FT{h} = \intinf e^{-a\abs{t}} e^{-ikt} \ dt
\end{equation*}
\textbf{Proof}
\begin{align*}
    \FT{h} &= \intinf e^{-a\abs{t}} e^{-ikt} \ dt \\
    &= \intinf e^{-a\abs{t}-ikt} \ dt \\
    &=\int^\infty_0 e^{-(ik-a)t} \ dt +\int_{-\infty}^0 e^{-(a-ik)t} \ dt \\
    &=\left [ \frac{1}{a-ik}e^{-(ik-a)t} \right ]^\infty_0 
    + \left [ \frac{1}{-a-ik}e^{-(a+ik)t} \right ]^0_{-\infty} \\
    &= \frac{1}{a-ik} + \frac{1}{a+ik}\\
    &= \frac{2a}{a^2+k^2}
\end{align*}
\end{document}
